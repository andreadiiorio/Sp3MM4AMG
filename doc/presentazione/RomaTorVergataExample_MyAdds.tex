% Copyright 2003--2007 by Till Tantau
% Copyright 2010 by Vedran Mileti\'c
%
% This file may be distributed and/or modified
%
% 1. under the LaTeX Project Public License and/or
% 2. under the GNU Free Documentation License.
%
% See the file doc/licenses/LICENSE for more details.

% $Header: /Users/joseph/Documents/LaTeX/beamer/doc/themeexamples/beamerthemeexample.tex,v 5a25f58600c3 2010/04/27 12:17:50 rivanvx $


\documentclass[aspectratio=169]{beamer}

\usetheme{RomaTorVergata}

\beamertemplatetransparentcovered	%multi step visibili ma in non evidenziati

\usepackage{times}
\usepackage[T1]{fontenc}
\usepackage[export]{adjustbox}	%figure align extra keys

\title{Sp3MM for AMG}
\subtitle{Sparse Triple Matrix Matrix Multiplication\\for\\AlgebraicMultiGrid}

\author{Andrea Di Iorio}
\institute{Università di Roma Tor Vergata}
\date{22/04/2022}

\begin{document}

\begin{frame}
	\titlepage
\end{frame}

\begin{frame}	{Outline}
	\tableofcontents
\end{frame}

\section{Results}
\subsection{Proof of the Main Theorem}

\begin{frame}<1-3>
	\frametitle{Proof That There Is No Largest Prime Number}
	\framesubtitle{A proof using \textit{reductio ad absurdum}.}

	\begin{theorem}
		There is no largest prime number.
	\end{theorem}
	\begin{proof}
		\begin{enumerate}
		\item<1-> Suppose $p$ were the largest prime number.
		\item<2-> Let $q := 1 + \prod_{i=1}^p i = 1+p!$.
		\item<3-> Then $q$ is not divisible by any $p' \in \{1,\dots,p\}$.
		\item<1-> Thus $q>p$ is also prime.\qedhere
		\end{enumerate}
	\end{proof}
\end{frame}

%%% my tests
\section{MyTests}
\subsection{columns}
\subsubsection{figures}
\begin{frame}
	\frametitle{Proof That There Is No Largest Prime Number-pauses}
	\framesubtitle{A proof using \textit{reductio ad absurdum}.}

	\begin{theorem}
		There is no largest prime number.
	\end{theorem}
	\begin{proof}
		\begin{enumerate}
		\item Suppose $p$ were the largest prime number.
		\pause
		\item Let $q := 1 + \prod_{i=1}^p i = 1+p!$.
		\item Then $q$ is not divisible by any $p' \in \{1,\dots,p\}$.
		\pause
		\item Thus $q>p$ is also prime.\qedhere
		\end{enumerate}
	\end{proof}
\end{frame}

\begin{frame} {columns figures}	
\begin{columns}
	\column{0.65\textwidth}
		\begin{itemize}
			\item	item 0
			\item	
				item1, newline tabbed
			\item	
			item2, newline
			\item item3, space only
			\item item4, space only
			and newline
			\item item5, space only
				and newline tabbed
		\end{itemize}
	\column{0.45\textwidth}
  		\includegraphics[width=.88\linewidth]{svgImg.svg.pdf}
  		\includegraphics[width=.88\linewidth]{svgImg.svg.pdf}
		%\begin{figure}[h!]  
    	%    %\includegraphics[scale=]{svgImg.pdf}
  		%	\includegraphics[width=.96\columnwidth,keepaspectratio,left]{svgImg.svg.pdf}
    	%    \caption{svg scaled at XXX\% of avail column (.45 slide)}
    	%\end{figure}
\end{columns}
\end{frame}

\end{document}
